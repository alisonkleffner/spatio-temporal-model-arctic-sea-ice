% !TeX program = pdfLaTeX
\documentclass[12pt]{article}
\usepackage{amsmath}
\usepackage{graphicx,psfrag,epsf}
\usepackage{enumerate}
\usepackage{natbib}
\usepackage{textcomp}
\usepackage[hyphens]{url} % not crucial - just used below for the URL
\usepackage{hyperref}

%\pdfminorversion=4
% NOTE: To produce blinded version, replace "0" with "1" below.
\newcommand{\blind}{0}

% DON'T change margins - should be 1 inch all around.
\addtolength{\oddsidemargin}{-.5in}%
\addtolength{\evensidemargin}{-.5in}%
\addtolength{\textwidth}{1in}%
\addtolength{\textheight}{1.3in}%
\addtolength{\topmargin}{-.8in}%

%% load any required packages here



% tightlist command for lists without linebreak
\providecommand{\tightlist}{%
  \setlength{\itemsep}{0pt}\setlength{\parskip}{0pt}}




\begin{document}


\def\spacingset#1{\renewcommand{\baselinestretch}%
{#1}\small\normalsize} \spacingset{1}


%%%%%%%%%%%%%%%%%%%%%%%%%%%%%%%%%%%%%%%%%%%%%%%%%%%%%%%%%%%%%%%%%%%%%%%%%%%%%%

\if0\blind
{
  \title{\bf A Spatio-Temporal Model for Arctic Sea Ice}

  \author{
        Alison Kleffner 1 \\
    Department of Statistics, University of Nebraska - Lincoln\\
     and \\     Susan VanderPlas 2 \\
    Department of Statistics, University of Nebraska - Lincoln\\
     and \\     Yawen Guan 3 \\
    Department of Statistics, University of Nebraska - Lincoln\\
      }
  \maketitle
} \fi

\if1\blind
{
  \bigskip
  \bigskip
  \bigskip
  \begin{center}
    {\LARGE\bf A Spatio-Temporal Model for Arctic Sea Ice}
  \end{center}
  \medskip
} \fi

\bigskip
\begin{abstract}
Arctic Sea Ice is a barrier between the warm air of the ocean and the
atmosphere, thus playing an important role in the climate. When narrow
linear cracks (leads) form in the sea ice, the heat from the ocean is
then released into the atmosphere. To estimate where cracks may form,
motion data from the RADARSAT Geophysical Processing System (RGPS) are
analyzed. The RGPS provides a set of trajectories (or cells) to trace
the displacements of sea ice, however, chunks of data are missing due to
the data collection method. We propose a spatial clustering and
interpolation method that allows us to infer missing observations and
estimate, where a crack may form. To do this feature inputs were created
for KNN clustering by creating a bounding box around each trajectory,
resulting in trajectories being assigned a cluster. A crack is
considered to have formed on the boundary between different clusters.
Within the clusters, spatiotemporal interpolation method is used to
infer missing locations. Our clustering approach is then compared to
other methods to determine ice crack formation, and cross-validation is
used to assess our interpolation method.
\end{abstract}

\noindent%
{\it Keywords:} spatial clustering, non-stationary, Gaussian process.
\vfill

\newpage
\spacingset{1.45} % DON'T change the spacing!

\hypertarget{introduction}{%
\section{Introduction}\label{introduction}}

\hypertarget{methods}{%
\section{Methods}\label{methods}}

\hypertarget{simulation}{%
\section{Simulation Study}\label{simulation}}

\hypertarget{results}{%
\section{Results}\label{results}}

\bibliographystyle{agsm}
\bibliography{bibliography.bib}


\end{document}
